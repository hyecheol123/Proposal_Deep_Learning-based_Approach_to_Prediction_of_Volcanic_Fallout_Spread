\documentclass{article}

% if you need to pass options to natbib, use, e.g.:
%     \PassOptionsToPackage{numbers, compress}{natbib}
% before loading neurips_2019

% ready for submission
% \usepackage{neurips_2019}

% to compile a preprint version, e.g., for submission to arXiv, add add the
% [preprint] option:
%     \usepackage[preprint]{neurips_2019}

% to compile a camera-ready version, add the [final] option, e.g.:
%     \usepackage[final]{neurips_2019}

% to avoid loading the natbib package, add option nonatbib:
\usepackage[nonatbib, preprint]{neurips_2019}

\usepackage[utf8]{inputenc}           % allow utf-8 input
\usepackage[T1]{fontenc}              % use 8-bit T1 fonts
\usepackage{hyperref}                 % hyperlinks
\usepackage{url}                      % simple URL typesetting
\def\UrlBreaks{\do\/\do-}             % To properly break the long url
\usepackage{booktabs}                 % professional-quality tables
\usepackage{amsfonts}                 % blackboard math symbols
\usepackage{nicefrac}                 % compact symbols for 1/2, etc.
\usepackage{microtype}                % microtypography

\usepackage{setspace}                 % Double Spacing
\usepackage{lmodern}                  % Font-size setting
\usepackage{graphicx}                 % For Figures
\usepackage[labelfont=bf]{caption}    % For Figures' Caption
\usepackage[numbers]{natbib}          % For Citation

\title{Deep Learning-based Approach \linebreak%
       to Prediction of Volcanic Fallout Spread
      }

% The \author macro works with any number of authors. There are two commands
% used to separate the names and addresses of multiple authors: \And and \AND.
%
% Using \And between authors leaves it to LaTeX to determine where to break the
% lines. Using \AND forces a line break at that point. So, if LaTeX puts 3 of 4
% authors names on the first line, and the last on the second line, try using
% \AND instead of \And before the third author name.

\author{
  Hyecheol Jang \\
  Department of Computer Sciences\\
  University of Wisconsin–Madison\\
  Madison, WI 53706 \\
  \texttt{hyecheol.jang@wisc.edu} \\

  \And
  Kangwook Lee\thanks{Faculty Advisor} \\
  Department of Electrical and Computer Engineering \\
  University of Wisconsin–Madison\\
  Madison, WI 53706 \\
  \texttt{kangwook.lee@wisc.edu} \\
}

\begin{document}

\maketitle

\begin{abstract}
  \fontsize{11pt}{11pt} \selectfont {
    The volcanic fallout ash is one of the major causes which make the volcanic eruption a fatal 
    natural disaster. The prediction on the spread of volcanic fallout can be made. However, the 
    calculation is complicated and requires huge computing resources to make an accurate prediction. 
    Previously, researchers figured out several ways to predict it faster, but it still required to 
    run the simulation. Therefore, we propose the use of deep learning to make a fast prediction of 
    the ash spread. Specifically, we will use cGAN based architecture to predict ash spread patterns
    from various input data including weather maps and eruption scale parameters. At the end of the 
    project, we plan to present a paper at a conference as well as open-source our trained model so 
    that the whole society can be well-prepared form the fatal disaster by getting accurate 
    predictions faster.
  }
\end{abstract}

\begin{doublespacing}
\section{Introduction} % TODO Review Requested
\fontsize{11pt}{11pt} \selectfont {
  The volcanic eruption is one of the most severe natural disasters which occasionally happens.
  However, compared to the other disasters which can be predicted, like typhoons and storms,
  predicting the exact time and scale of volcanic eruption considered as a very difficult task, 
  which might never be accurate, according to 
  Einarsson~\cite[as cited in][para.7 \& 30]{fountain_2015}. 
  Although the volcanic eruptions cause countless casualties and massive economic loss in various 
  ways, according to Yun~\cite[p.274]{Yun_2013}, the fallout ash is considered as one of the most 
  significant by-products causing deadly effects. Resulting in the respiratory damage of lives, 
  threatening the safety and reliability of air transportation, and the collapse of structures 
  caused by sedimentation of the ash, the fallout ash causes various impairments in a wide range of 
  areas~\cite[p.274-275]{Yun_2013}.
  
  To mitigate the hazardous effects caused by the fallout ash, it is important to properly predict 
  the direction of ash spread and alert citizens who in the affected area. Historically, according 
  to Bonadonna et al. \cite[p.3-4]{Bonadonna2012}, scientists tried to make Volcanic Ash and 
  Dispersion/Tracking Model (VATDM) by using the Eulerian or the Lagrangian approach. These models 
  usually focused on predicting the pathway of fallout ash dispersion and the amount of deposited 
  ash while accepting limited variables~\cite[p.276]{Yun_2013}.

  However, these models’ predictability might be limited due to the computing 
  performance~\cite[p.745-746]{Tanaka2022}, as the model needs to calculate the movement of all 
  particles that need to be predicted. To be more specific, there is a positive relationship between 
  the number of particles to be tracked by VATDM and the calculation requirement. Scollo et al. 
  \cite{SCOLLO2011129} confirmed that it required up to hundreds of millions of particles to get 
  accurate simulation results, meaning that the simulation with a small number of particles might 
  not be able to illustrate the region of dispersal accurately.

  To make a faster prediction, researchers studied on making a lighter and faster prediction model
  \cite{Searcy1998}, standardizing the eruption parameters required to run the model 
  \cite[p.7]{Webley2009}, and figuring out the way to store and retrieve different kinds of data
  (satellite and prediction from the mathematical model), needed when deciding the final 
  prediction result more efficiently~\cite{Sorokin2016}. Though the models themselves and 
  utilization efficiency of simulated results have been improved with the effort of numerous 
  scholars, still the needs for the calculation process remain, which yet limit the ability of 
  prediction by the computing power.

  In this research, we are going to suggest a faster way to get the prediction on the pathway and 
  area that might be affected by volcanic fallout ash: deep learning. By utilizing the
  state-of-the-arts deep learning architectures, our goal includes verifying whether deep learning 
  can predict the spread of fallout ash. (See \textbf{Figure \ref{fig1}} for the model's structure)

  \fontsize{10pt}{10.5pt} \selectfont{
    \begin{figure}
      \centering
      \includegraphics[width = \linewidth]{ModelStructure.png}
      \caption{\textbf{Overview of the Research Goal}
                This research aims to make a deep-learning model predicting volcanic fallout 
                spread while accepting the weather maps (both surface and upper air) and eruption 
                scale parameters (Plume height, Plume shape, Volume of Emitted Ash, and the location
                of the mountain) as an input.
              }
      \vspace{-5mm}
      \label{fig1}
    \end{figure}
  }

  To be more specific, we are going to construct a model based on the conditional Generative 
  Adversarial Network (cGAN), proposed by Isola et al. \cite{isola2016imagetoimage}, which focused 
  on translating one image to the other type of image (Image-to-Image Translation). Comparing to the 
  previous work \cite{isola2016imagetoimage}, while the previous architecture only got an image as 
  an input and produced one output image, our deep learning model should get several images and need
  to produce continuous images representing the prediction of the direction and speed of the ash 
  spread. The details regarding the model structure are written below (Method - Model structure).
}

\section{Methods}  % TODO Review Requested
\fontsize{11pt}{11pt} \selectfont {
  \paragraph{Data Collection}
  To achieve our goal, making a generator which predicts the possible area of ash spread for the 
  given weather conditions and eruption scale, we first need to collect input and output data pairs.
  To collect these, for the input images, we will download both surface and upper air weather maps 
  from the “Meteorological Data Open Portal” operated by the Korea Meteorological Administration 
  (KMA)~\cite{MDOP}. To get the output images, we will go through the traditional procedure 
  (simulating by VATDM).
  
  The weather prediction data will be obtained by the same location where we found the weather maps:
  download prediction made by RDAPS(regional data assimilation and prediction system)~\cite{MDOP}.
  The simulation parameters except for the weather data will be set manually. To represent various 
  eruption cases, for each mountain to be used as the sample, we will simulate with the constants 
  representing Volcanic Explosivity Index (VEI) 3 (Moderate Large Eruption), VEI 5 (Very Large 
  Eruption), and VEI 7 (Massive Explosive Eruption)~\cite[p.1232]{Newhall1982}. 

  \paragraph{Model structure}
  We decide to use a conditional GAN-based model, which has been shown to have a good performance on 
  generating indistinguishable images compare to the \emph{real} images 
  \cite{isola2016imagetoimage}. The GAN has two distinctive parts of functions - Generator and 
  Discriminator - and we need to fit both functions simultaneously. The goal for the Discriminator 
  is to check whether the generated image is fake or not, while the Generator's goal is to make an 
  image, which is enough to make the Discriminator categorize it as a \emph{real} image. While the 
  original GAN only uses a random noize to generate the output, the conditional GAN gets not only 
  random noize but also the \emph{condition} vectors as an input. For our model, the condition 
  vector contains information about the weather maps and the eruption scale.

  Comparing our model with Isola et al.\cite{isola2016imagetoimage}'s cGAN model, while Isola et 
  al. provided only one image as an input, we have to pass the several weather map images and the 
  numerical vector indicating the eruption scale of the volcano. Moreover, while Isola et 
  al.\cite{isola2016imagetoimage} challenged to provide randomness to the output, we want our model 
  to provide the identical output when the same inputs have been provided. Lastly, our model needs 
  to provide continuous images to represent the dispersion of volcanic ash over time, however, the 
  Isola et al.\cite{isola2016imagetoimage}'s model only needed to generate a still photograph.

  To construct the model, we will use the U-Net architecture, proposed by Ronneberger et al.
  \cite{ronneberger2015unet}, for the convolutions passing the information on the weather maps to 
  the model. As the U-Net architecture has achieved promising results on several different
  image tasks\cite{isola2016imagetoimage,james2018simtoreal}, we expect that the architecture will 
  also work for our task. However, we still have to find a solution to pass information from several
  different images and one numerical vector to our neural network when we train our model. After 
  that, we are going to choose proper loss functions that make the best result in predicting the 
  relationship between elements depicted on the weather maps to the spread of the ash.

  \paragraph{Model Analysis and Experiments}
  While we maneuver over various loss functions, we are going to make various models utilizing 
  different loss functions and compare those with the quantitative metric. As indicated in the 
  previous paper \cite{isola2016imagetoimage}, it is difficult to say that there exists one best 
  representation of the quality of translated images. Therefore, we need to set reliable 
  measurements before we analyze models using different loss functions.

  Once we decide the model structure, we need to determine the number of training datasets so that 
  it can translate the weather map images to simulation results flawlessly. To do these tasks, we 
  need to train models with different amounts of inputs. While we select training sets randomly, we 
  should ensure that we pick a combination of data that represents the dynamic climate of Korea as 
  it experiences four distinctive seasons. After we fit the model with a different number of 
  training datasets, we will investigate how well each model draws the result.
}

\section{Timeline} % TODO Review Requested
\fontsize{11pt}{11pt} \selectfont {
  The first thing we should do is to get the input. There is no available API to get the weather 
  maps and data from KMA. Therefore, we must use the interactive website provided by KMA to download
  the data; this is expected to take some time. Moreover, the calculation time for that simulation 
  was not ignorable, according to a previous experiment (it takes approximately one hour to get one 
  output through a laptop with Intel’s i5-3320M CPU). Considering that we have access to better 
  computing power than the previous test, we expect to finish data collection within two months 
  after the project's initiation.

  The next step is to train the generator with a given input-output pair. As we have to fit the 
  conditional GAN based model and verify the result with various test cases, we expect to take two 
  additional months. For the last month of this research, we will write a report and create posters.

  The detailed timeline is in \textbf{Table \ref{table1}}.

  \fontsize{10pt}{10pt} \selectfont {
    \begin{table}
      \caption{Detailed Timeline}
      \label{table1}
      \centering
      \begin{tabular}{lll}
        \toprule
        \cmidrule(r){1-2}
        Due date     & Task description \\
        \midrule
        Mar. 14 & Select proper VATDM to generate training/testing sets \\ 
        Mar. 20 & Collect weather maps and weather prediction data \\
        Apr. 10 & Run simulation and collect outputs \\
        Apr. 31 & Choose candidates for final model's structure \\
        Jun. 15 & Conduct and analyze experiments on each nodels \\
        Jul. 15 & Write reports \\
        \bottomrule
      \end{tabular}
    \end{table}
  }
}

\section{Conclusion and Future Direction} % TODO Review Requested
\fontsize{11pt}{11pt} \selectfont {
  We expect to make a deep neural network model to predict possible pathways and areas where 
  volcanic fallout ash spreads faster than the previous methods by using simpler and lighter input, 
  the weather maps. Not only does the model reduce the computing time to predict the spread, but the
  model also aims to reproduce results as similar as possible compared to the original training 
  result. To verify our idea and model structure, we utilize weather maps representing the weather 
  conditions and volcanos around the Korean Peninsula.
}

\end{doublespacing}

\medskip

\bibliography{reference}
\bibliographystyle{apalike}


\end{document}
