\documentclass{article}

% if you need to pass options to natbib, use, e.g.:
%     \PassOptionsToPackage{numbers, compress}{natbib}
% before loading neurips_2019

% ready for submission
% \usepackage{neurips_2019}

% to compile a preprint version, e.g., for submission to arXiv, add add the
% [preprint] option:
%     \usepackage[preprint]{neurips_2019}

% to compile a camera-ready version, add the [final] option, e.g.:
%     \usepackage[final]{neurips_2019}

% to avoid loading the natbib package, add option nonatbib:
\usepackage[nonatbib, preprint]{neurips_2019}

\usepackage[utf8]{inputenc} % allow utf-8 input
\usepackage[T1]{fontenc}    % use 8-bit T1 fonts
\usepackage{hyperref}       % hyperlinks
\usepackage{url}            % simple URL typesetting
\def\UrlBreaks{\do\/\do-}   % To properly break the long url
\usepackage{booktabs}       % professional-quality tables
\usepackage{amsfonts}       % blackboard math symbols
\usepackage{nicefrac}       % compact symbols for 1/2, etc.
\usepackage{microtype}      % microtypography

\usepackage{setspace} % Double Spacing
\usepackage{lmodern} % Font-size setting

\usepackage[numbers,round]{natbib}

\title{Generating Model to Predict the Spread of Volcanic Fallout Ash Spread%
  by using Image-to-Image Translation of Weather Map%
  }

% The \author macro works with any number of authors. There are two commands
% used to separate the names and addresses of multiple authors: \And and \AND.
%
% Using \And between authors leaves it to LaTeX to determine where to break the
% lines. Using \AND forces a line break at that point. So, if LaTeX puts 3 of 4
% authors names on the first line, and the last on the second line, try using
% \AND instead of \And before the third author name.

\author{
  Hyecheol Jang \\
  Department of Computer Sciences\\
  University of Wisconsin–Madison\\
  Madison, WI 53706 \\
  \texttt{hyecheol.jang@wisc.edu} \\

  \And
  Kangwook Lee\thanks{Faculty Advisor} \\
  Department of Electrical and Computer Engineering \\
  University of Wisconsin–Madison\\
  Madison, WI 53706 \\
  \texttt{kangwook.lee@wisc.edu} \\
}

\begin{document}



\maketitle

\begin{abstract}
  \fontsize{11pt}{11pt} \selectfont {
    The abstract paragraph should be indented \nicefrac{1}{2}~inch (3~picas) on
    both the left- and right-hand margins. Use 10~point type, with a vertical
    spacing (leading) of 11~points.  The word \textbf{Abstract} must be centered,
    bold, and in point size 12. Two line spaces precede the abstract. The abstract
    must be limited to one paragraph.
  }
\end{abstract}

\begin{doublespacing}
\section{Introduction} % TODO Review Requested
\fontsize{11pt}{11pt} \selectfont {
  The volcanic eruption is one of the most severe natural disasters which occasionally happens.
  However, compares to the other disasters which can be predicted, like typhoons and storms,
  predicting the exact time and scale of volcanic eruption considers as a very difficult task which 
  might never be accurate, according to Einarsson~\citep[as cited in][para 7 \& 30]{fountain_2015}. 
  Though there exist various ways that volcanic eruption causing countless casualties and massive 
  economic loss, according to Yun~\citet[p. 274]{Yun_2013}, the fallout ash, the focus of this 
  research, considers as one of the most significant by-products making deadly effects.
  Including the respiratory damage of lives, threatening the safety and reliability of air 
  transportation, and the collapse of structures caused by sedimentation of the ash, the fallout ash
  causing various impairments in a wide range of areas~\citep[p. 274-275]{Yun_2013}.
  
  To mitigate the hazardous effects caused by the fallout ash, it is important to properly predict 
  the direction of ash spread and evacuate or alert citizens who reside through the path of the ash 
  dispersion. Historically, by using the Eulerian or the Lagrangian 
  approach~\citet[p. 3-4]{Bonadonna2012}, scientists are tried to make Volcanic Ash and 
  Dispersion/Tracking Model (VATDM). These models usually focus on predicting the pathway of fallout
  ash dispersion and the amount of deposited ash while accepting limited 
  variables~\citep[p. 276]{Yun_2013}.

  However, these models’ predictability might be limited due to computing 
  performance~\citep[p. 745-746]{Tanaka2022}. To make a faster prediction, the fellow scholars 
  conducted researches on making a lighter and faster prediction model~(\citet{Searcy1998}), 
  standardizing the eruption parameters required for running the model~(\citet[p. 7]{Webley2009}), 
  and figuring out the way to store and retrieve different kinds of data (satellite and prediction 
  from the mathematical model) more efficiently~(\citet{Sorokin2016}). Though the advance of the 
  models' and predict utilizations' efficiency has been improved with the effort of numerous 
  scholars, still the needs of the calculation process last, which yet limiting the ability of 
  prediction to the computing power.

  In this research, we are going to suggest a faster way to get the prediction on the pathway and 
  area that might be affected by volcanic fallout ash: Image-to-Image Translation. By utilizing the
  state-of-the-arts deep neural network, our goal includes verifying whether conditional GAN based 
  deep neural network can extract the relationship between the flow of air (represented by weather 
  maps) and the spread of fallout ash.
}

\section{Methods}  % TODO Review Requested
\fontsize{11pt}{11pt} \selectfont {
  \paragraph{Data Collection}
  To achieve our goal, making generator which predicts the possible area of ash spread for the given
  pair of weather maps, we first need to collect input and output data pairs. To collect these, for 
  the input image, we will download those from the “Meteorological Data Open Portal” operated by the
  Korea Meteorological Administration (KMA)~\citep{MDOP}. To get the output data that will be paired
  with the weather maps, we will go through the traditional procedure with either FALL3D, or 
  Puff-UAF model.
  
  The simulation parameters except for the weather data – plume height, number of particles to be 
  predicted, and the mountain that has been erupted – will be set manually. To represent various 
  eruption cases, for each mountain to be used as sample, we will simulate with the constants 
  representing Volcanic Explosivity Index (VEI) 3 (Moderate Large Eruption), VEI 5 (Very Large 
  Eruption), and VEI 7 (Massive Explosive Eruption)~\citep[p. 1232]{Newhall1982}. The remaining, but
  the most important, parameter for the simulation model, weather prediction data, will be obtained 
  by the same location where we found the weather maps: download prediction made by RDAPS(regional 
  data assimilation and prediction system)~\citep{MDOP}. These datasets are extremely huge: one day 
  of weather data takes 1.32 GigaBytes, and we are thinking of getting at least 5 years of data for 
  both training and testing sets. We need to find a reliable storage to save all the weather data, 
  weather maps, and output images.

  \paragraph{Model structure}
  Starting from conditional GAN, which has been shown to have better performance on generating 
  indistinguishable images compare to the "real" images~\citet{isola2016imagetoimage}, we are going 
  to choose proper loss function which has the best result on predicting the relationship between 
  elements depicted on the weather maps to the spread of the ash. For the convolutions, we will use
  the U-Net architecture proposed by Ronneberget et al.~\citet{ronneberger2015unet}, as the
  architecture has been achieved a promising result on several different imaging 
  tasks~\citep{isola2016imagetoimage, james2018simtoreal}.

  \paragraph{Model Analysis and Experiments}
  While we mauver over various loss functions, we are going to make various models utilizing 
  different loss functions and compare those with the quantitative metric. As indicated in the 
  previous paper~\citet{isola2016imagetoimage}, it is difficult to say there exist one best 
  representation on the quality of translated images. Therefore, we need to set a realiable 
  measurements before we analyze models using different loss functions.

  Once we decide the model structure, we need to verify the amount of sufficient size of training 
  datasets so that it can translate the weather map image to simulation result flawlessly. To do 
  these tasks, we need to train models with different amounts of inputs. As Korean Peninsula 
  experiencing four distinctive seasons, while we picking up the training set randomly, we would 
  better ensure that we pick a combination of data that represents the dynamic climate of Korea. 
  After we fit the model with different number of training datasets, we will investingate how well 
  each model draw the result.
}

\section{Timeline} % TODO Review Requested
\fontsize{11pt}{11pt} \selectfont {
  The first thing we should do is to get the input. As there is no available API to get the weather 
  maps and data from KMA, but they provide an interactive website to download the data, it might 
  take some time to get all data manually. Moreover, the calculation time for that simulation was 
  not ignorable, according to previous experiment (it takes approximately one hour to get one output
  with the laptop having Intel’s i5-3320M CPU). Considering the factor that we have access to better
  computing power than the previous test, we are expecting to finish data collection within two 
  months after project initiated.

  The next step is to train the generator with given input-output pair. To fit conditional GAN based
  model and to verify the result with various test cases, we are expecting to take two additional 
  months here. For the last month of this research, we will write a report and posters.
}

\section{Conclusion and Future Direction} % TODO Review Requested
\fontsize{11pt}{11pt} \selectfont {
  We are expected to have a faster way to predict possible pathway and area of volcanic fallout ash 
  spread by using simpler and lighter input, weather maps, compare to the weather dataset predicted 
  by RDAPS. Not only reduce the computing time to predict the spread, but the model aims to
  reproduce as same result as possible compares to the original training result.

  Because of the weather maps’ style is different with the agencies that produce the map (NOAA’s way
  of making the map and KMA’s way of making the map is different), our model was not focused to 
  produce an output from the maps that has been publicized by different agencies. Moreover, because 
  of the limitation of dataset, we are only able to train and test on the area near Korean 
  Peninsula; the models could not be verified on the different regional settings.
}

\end{doublespacing}

\medskip

\bibliography{reference}
\bibliographystyle{apalike}


\end{document}
